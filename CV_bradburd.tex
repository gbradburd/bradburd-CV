%COMPILE WITH LuaLaTeX
\documentclass{article}
\usepackage[charter]{mathdesign}
%\usepackage{times}
\usepackage{xcolor}
\usepackage{hyperref}
\usepackage{array}
\hypersetup{colorlinks,urlcolor=blue}
\usepackage[margin=1in]{geometry}
\usepackage{setspace}
\usepackage{fontspec}
\setmainfont{Corbel}
\usepackage{breakurl}
\newcolumntype{R}{>{\raggedleft\arraybackslash}p{0.295\textwidth}}
\setlength{\parindent}{0in}
\usepackage[bottom]{footmisc}
\usepackage{tabularx}
\usepackage{array}
\begin{document}
\pagenumbering{gobble}
%
\begin{huge}
\bf{Gideon Bradburd}\
\end{huge}
\vspace{-0.25cm}
\\
%
\vspace{-0.7cm}
\rule{470pt}{0.4pt}
%\noindent\makebox[\linewidth]{\rule{18cm}{0.4pt}}
\vspace{0.3cm}
%
\\
3348 Storer Hall \hfill email: \href{mailto:gbradburd@ucdavis.edu}{gbradburd@ucdavis.edu}\\
Evolution and Ecology, UC Davis \hfill phone: 203-687-5224\\
One Shields Ave, Davis, CA 95616 \hfill website: \href{http://genescape.org}{genescape.org}\\
\vspace{-0.6cm}
%
\section*{EDUCATION}
\vspace{-0.6cm}
\rule{470pt}{0.4pt}
%
\begin{tabular}{@{}p{0.05\textwidth}p{0.6\textwidth}R@{}}
\\
\bf{PhD} & \it{University of California, Davis}  & Sept. 2009 - Jun. 2015 (expected)\\
 & \hspace{0.5cm}Population Biology Graduate Group & \\
 & \hspace{0.5cm}Department of Ecology and Evolutionary Biology & \\
 & \hspace{0.5cm}Advisors: Graham Coop, Brad Shaffer & \\ 
%
\hfill\\
%
\textbf{BS} & \it{Yale University} & Sept. 2004 - May 2008 \\
& \hspace{0.5cm}Ecology and Evolutionary Biology & \\
& \hspace{0.5cm}Honors Thesis Advisor: Tom Near & \\
\end{tabular}
%
\section*{PUBLICATIONS}
\vspace{-0.6cm}
\rule{470pt}{0.4pt}
%

\begin{tabular}{>{\everypar{\hangindent1cm}}p{0.95\textwidth}p{0.05\textwidth}}
\hfill\\
\underline{In review} \hfill\\
\vspace{-0.1cm}
\textbf{Bradburd, G.S.}, P.L. Ralph, and G.M. Coop. A spatial framework for understanding population structure and admixture.  %\newline
 On bioRxiv at \href{http://dx.doi.org/10.1101/013474}{http://dx.doi.org/10.1101/013474}. (PLoS Genetics)& \hfill\\
%
\vspace{0.2cm}
%
\underline{Published and Accepted Papers} \hfill\\
\vspace{-0.1cm}
Agrawal, A.A., A.P. Hastings, \textbf{G.S. Bradburd}, E.C. Woods, T. Z{\"u}st, J. Harvey, T. Bukovinszky. (\textit{accepted}). 
Evolution of plant growth and defense in a continental introduction. \underline{American Naturalist}.\\
%
\vspace{-0.1cm}
Wang, I.J.* and \textbf{G.S. Bradburd}.* (2014). Isolation by Environment. \underline{Molecular Ecology} 23: 5649-5662.  & \hfill\\
\hspace{4.5mm} *denotes equal authorship&\\
%
\vspace{-0.1cm}
%
\textbf{Bradburd, G.S.}, P.L. Ralph, and G.M. Coop. (2013). Disentangling the effects of geographic and ecological isolation on genetic differentiation. \underline{Evolution} 67: 3258-3273. & \hfill \\
%
\vspace{-0.1cm}
%
Rejmanek, D, P. Freycon, \textbf{G.S. Bradburd}, J. Dinstell, and J. Foley.  (2013). Unique strains of \textit{Anaplasma phagocytophilum} segregate among diverse questing and non-questing \textit{Ixodes} tick species in the western United States.  \underline{Ticks and Tick-borne Diseases} 4: 482-487. & \hfill\\
%
\vspace{-0.1cm}
%
Rejmanek, D., \textbf{G.S. Bradburd}, and J. Foley.  (2012). Molecular characterization reveals distinct genospecies of \textit{Anaplasma phagocytophilum} from diverse North American hosts.	\underline{Journal of Medical Microbiology} 61: 204-212. & \hfill \\
%
\vspace{-0.1cm}
%
Near, T.J., C.M. Bossu, \textbf{G.S. Bradburd}, R.L. Carlson, R.C. Harrington, P.R. Hollingsworth Jr., B.P. Keck, D.A. Etnier.  (2011). Phylogeny and temporal diversification of darters (\textit{Percidae: Etheostomatinae}).  \underline{Systematic Biology} 60: 565-595& \hfill\\
%
\end{tabular}

\section*{GRANTS AND FELLOWSHIPS}
\vspace{-0.6cm}
\rule{470pt}{0.4pt}
%
\begin{tabular}{>{\everypar{\hangindent1cm}}p{0.85\textwidth}p{0.1\textwidth}}
\hfill\\
\textbf{NSF Doctoral Dissertation Improvement Grant (DDIG)}, National Science Foundation (\$13,918) & \hfill 2014-15\\
\textbf{XSEDE SuperComputing Resource Allocation}, XSEDE (50,000 SUs) & \hfill 2014-05\\
\textbf{Center for Population Biology Research Fellowship} CPB, UC Davis (\$4,000) & \hfill 2011-13\\
%\textbf{Center for Population Biology Research Fellowship} CPB, UC Davis (\$2,000) & \hfill 2011-12\\
\textbf{NSF Graduate Research Fellow} National Science Foundation (\$90,000) & \hfill 2010-12\\
\textbf{Graduate Scholars Fellowship} in Population Biology, UC Davis (\$30,000) & \hfill  2009-10\\
%\hspace{4.5mm}Awarded in recognition of outstanding academic record and promise of productive scholarship\\
\textbf{Environmental Internship Award} Yale University (\$6,000) & \hfill 2006-7\\ 
\textbf{Richter Summer Travel Fellowship} Yale University (\$1,000)  & \hfill 2007\\
\textbf{Mellon Forum Research Grant} Yale University (\$500) & \hfill 2007\\
%\textbf{Environmental Internship Award} Yale University (\$2,000) & \hfill 2006\\
\end{tabular}
%
%\newpage
%
\section*{TEACHING EXPERIENCE}
\vspace{-0.6cm}
\rule{470pt}{0.4pt}
%
\begin{tabular}{>{\everypar{\hangindent1cm}}p{0.60\textwidth}p{0.225\textwidth}p{0.1\textwidth}}
\hfill\\
EVE100 Introduction to Evolution & Teaching Assistant & \hfill 2015	\\
UCLA/La Kretz Center Workshop in Conservation Genomics & Instructor & \hfill 2014 \\
EVE103 Macroevolution, Speciation, and Phylogeny (Lecture, Lab) & Teaching Assistant & \hfill 2014 \\
Herpetology (Field, Lab, Lecture) & Teaching Assistant & \hfill 2012\\
Bodega Bay Applied Phylogenetics Workshop & Teaching Assistant & \hfill 2012-13\\
\end{tabular}
%
%
\section*{INVITED TALKS}
\vspace{-0.6cm}
\rule{470pt}{0.4pt}
%
\begin{tabular}{>{\everypar{\hangindent1cm}}p{0.85\textwidth}p{0.1\textwidth}}
\hfill\\
\textbf{Bradburd, G.S.}, P. Ralph, G. Coop. The Geography of Genetic Structure and Admixture. & \hfill 2015 \\
\hspace{1cm} Stanford University.\\
%
\textbf{Bradburd, G.S.}, P. Ralph, G. Coop. A new approach to the geography of genetic structure. & \hfill 2014 \\
\hspace{1cm} University of Southern California.\\
%
\textbf{Bradburd, G.S.}, P. Ralph, G. Coop. The spatial context of genetic admixture. & \hfill 2014 \\
\hspace{1cm} University of Montana.\\
%
\textbf{Bradburd, G.S.}, P. Ralph, G. Coop.  A spatial framework for studying genetic differentiation.  & \hfill 2014 \\
\hspace{1cm} University of Minnesota.\\
%
\textbf{Bradburd, G.S.} The effects of geographic and ecological isolation on genetic differentiation. & \hfill 2014 \\
\hspace{1cm} Sonoma State University. \\
%
\textbf{Bradburd, G.S.} \textit{BEDASSLING} your data: background and functions of the R package \textit{BEDASSLE}.  & \hfill 2013 \\
\hspace{1cm} University of Texas, Austin\\
\end{tabular}
%
\section*{CONTRIBUTED PRESENTATIONS}
\vspace{-0.6cm}
\rule{470pt}{0.4pt}
\begin{tabular}{>{\everypar{\hangindent1cm}}p{0.85\textwidth}p{0.1\textwidth}}
\hfill\\
\textbf{Bay Area Population Genomics (BAPG)} & \hfill 2014\\
\textbf{Society for Molecular Biology and Evolution} & \hfill 2014\\
\textbf{Society for the Study of Evolution} & \hfill 2013-14\\
\textbf{CPB Dept. Seminar} & \hfill 2011,13\\
\textbf{E\&EB Senior Symposium, Yale University} & \hfill 2008\\
\end{tabular}

\section*{SERVICE}
\vspace{-0.6cm}
\rule{470pt}{0.4pt}
\begin{tabular}{>{\everypar{\hangindent1cm}}p{0.8\textwidth}p{0.15\textwidth}}
\hfill\\
Student Representative, Population Biology Grad Group Steering Committee & \hfill 2013-14\\
%
Chair of Student Research Symposium, Center for Population Biology/Pop Bio Grad Group & \hfill 2013\\
%
Course Organizer, The Bodega Bay Applied Phylogenetics Workshop & \hfill 2012-13\\
%
Chair, UC Davis Non-Model Genomics Workshop Committee & \hfill 2010\\
%
Graduate Student Mentor & \hfill 2009-present\\
%
\end{tabular}
%
\newpage
\section*{REVIEWER}
\vspace{-0.6cm}
\rule{470pt}{0.4pt}
\\\\
Evolution, 
Methods in Ecology and Evolution,
Diversity and Distributions,
Integrative and Comparative Biology,
Bioinformatics, 
Nature - Scientific Reports,
Heredity,
Molecular Ecology

\section*{SOCIETY MEMBERSHIPS}
\vspace{-0.6cm}
\rule{470pt}{0.4pt}
\\\\
Society for the Study of Evolution (SSE); Society for Molecular Biology and Evolution (SMBE)

\section*{PREVIOUS APPOINTMENTS}
\vspace{-0.6cm}
\rule{470pt}{0.4pt}
\begin{tabular}{>{\everypar{\hangindent1cm}}p{0.8\textwidth}p{0.15\textwidth}}
\hfill\\
\textbf{Research Assistant, Prof. Thomas Near, Yale University} & \hfill 2008-09 \\ 
\hspace{4.5mm}Carried out research on phylogeography of North American fishes & \\
%
\textbf{Technician, Yale Peabody Museum of Natural History} & \hfill 2007-08 \\ 
\hspace{4.5mm}Identified and sorted vertebrate specimens&\\ 
%
\textbf{Lab Technician, Marine Biology Lab, Prof. Leo Buss} & \hfill 2005 \\ 
\hspace{4.5mm}Maintained populations of \textit{Hydractinia echinata} and \textit{Nematostella vectensis}&
\end{tabular}


\section*{FIELDWORK}
\vspace{-0.6cm}
\rule{470pt}{0.4pt}
\begin{tabular}{>{\everypar{\hangindent1cm}}p{0.8\textwidth}p{0.15\textwidth}}
\hfill\\
\textbf{Northern California} & \hfill 2010-2012\\
\hspace{4.5mm}Sampled tissue and ectoparasites from lizards and small rodents.\\
%
\textbf{American Southeast and Midwest} & \hfill 2008\\ 
\hspace{4.5mm}Sampled freshwater fish throughout the Mississippi Drainage.\\
%
\textbf{Costa Rica}  & \hfill  2007\\ 
\hspace{4.5mm}Collected reptiles and amphibians in the rainforest for the Yale Peabody Museum.\\ 
\hspace{4.5mm}Developed skills in collecting and preserving herpetological specimens.\\
\hspace{4.5mm}Monitored status of the Variable Harlequin Toad (\textit{Atelopus varius}).\\
%
\textbf{Suriname}  & \hfill  2006\\
\hspace{4.5mm}Collected birds in coastal and southern Suriname for the Yale Peabody Museum.\\ 
\hspace{4.5mm}Developed skills in collecting (mist-netting, shooting) and preserving ornithological specimens (anatomicals, skeletons, skins).\\ 
\end{tabular}

\end{document}



\section{\underline{LANGUAGES}}
\textbf{Spanish}: proficient speaking and reading\\
\textbf{Sranan Tongo (Taki-taki)}: fair, but only in the context of collecting specimens in the field

%\end{resume}


%GRAVEYARD
\textbf{Bradburd, G.S.}, P.L. Ralph, and G.M. Coop. Disentangling the effects of geographic and ecological isolation on genetic differentiation. Evolution 67(11):3258-3273.
%
Rejmanek, D, P. Freycon, \textbf{G.S. Bradburd}, J. Dinstell, and J. Foley.  Unique strains of \textit{Anaplasma phagocytophilum} segregate among diverse questing and non-questing \textit{Ixodes} tick species in the western United States.  Ticks and Tick-borne Diseases 4(6):482-487.
%
Rejmanek, D., \textbf{G.S. Bradburd}, and J. Foley.  2012.  Molecular characterization reveals	distinct genospecies of \textit{Anaplasma phagocytophilum} from diverse North American hosts.	Journal of Medical Microbiology. 61(2):204-212.
%
Near, T.J., C.M. Bossu, \textbf{G.S. Bradburd}, R.L. Carlson, R.C. Harrington, P.R. Hollingsworth	Jr., B.P. Keck, D.A. Etnier. 2011.  Phylogeny and temporal diversification of darters (\textit{Percidae: Etheostomatinae}).  Systematic Biology 60(5):565-595.


%\cfoot{}  
%\pagestyle{fancy} 
%\newenvironment{myindentpar}[1]%
%   {\begin{list}{}%
%             {\setlength{\leftmargin}{#1}}%
%             \item[]}
%    {\end{list}}

%\textbf{Relevant Coursework}
%\begin{myindentpar}{6mm}
%{UC Davis: Core Principles in Population Biology, Mathematical Methods in Population Biology, Macroevolution, Principles of Biological Data %Analysis}\vspace{3mm}

%Yale: General Chemistry (w/ lab), General Physics, General Ecology, Evolutionary Biology, Diversity of Life, Molecular Systematics (intensive lab), %Genetics, Conservation Genetics, Phylogenetic Methods and Approaches, Biology of Terrestrial Arthropods (w/ lab), Ornithology (w/ lab), %Introductory Geoscience, Petrology and Mineralogy, Natural Resources and Sustainability
%\end{myindentpar}

\textbf{Integrative Biology Fellowship} UT Austin, 2009-'10 (declined)\\ \vspace{0.3mm}
\hspace{4.5mm}Awarded to top applicants in the UT system\\
\textbf{Dorothy Clark Lettice Fellowship} Kansas University, 2009-'10 (declined)\\  \vspace{0.3mm}
\hspace{4.5mm}Awarded in recognition of high academic ability\\

%\section{\underline{VOLUNTEER AND LEADERSHIP EXPERIENCE}}
%\textbf{Yale Freshman Outdoor Orientation Trips (FOOT) Leader} 2005-'07\\ \vspace{0.3mm}
%\hspace{4.5mm}Led 4-6 day trips in the wilderness with 8-10 incoming Yale freshmen\\ \vspace{0.3mm}
%\hspace{4.5mm}Trained and certified in Wilderness First Aid and CPR\\
%\textbf{FOOT Public Relations Core Head} 2006-'07\\ \vspace{0.3mm}
%\hspace{4.5mm}Organized all public relations initiatives for the FOOT program, including sponsorship bids, new\\ \vspace{0.3mm} 
%\hspace{4.5mm}leader recruitment, and advertising to incoming freshmen\\ 
%\textbf{Volunteer Income Tax Assistance Site Coordinator, New Haven} 2005-'07\\ \vspace{0.3mm}
%\hspace{4.5mm}Coordinated an office and trained volunteers to administer Earned Income Tax Credit to local\\ \vspace{0.3mm}
%\hspace{4.5mm}individuals and families\\ 
%\textbf{Local Director of Client Services, National Student Partnerships, New Haven} 2004-'06\\ \vspace{0.3mm}
%\hspace{4.5mm}Worked with socioeconomically disadvantaged clients to navigate social bureaucracy and find\\ \vspace{0.3mm}
%\hspace{4.5mm}employment, affordable housing, and public benefits\\ \vspace{0.3mm}
%\textbf{Community Health Educator, New Haven} 2005-'06\\ \vspace{0.3mm}
%\hspace{4.5mm}Taught in local schools to provide health education, specifically in communication skills with\\ \vspace{0.3mm}
%\hspace{4.5mm}reference to substance and relationship abuse, nutrition, and sexual health\\
%\textbf{DEMOS Classroom Volunteer, Lincoln-Bassett Elementary School, New Haven} 2004-'05\\ \vspace{0.3mm}
%\hspace{4.5mm}Demonstrated basic scientific principles in a fun and accessible way\\ 
%\textbf{Bradburd, G.S.}, P.L. Ralph, G.M. Coop.   Disentangling the effects of geographic and ecological isolation on genetic differentiation.  Society for the Study of Evolution. 2013. Snowbird, Utah.\

%\textbf{Bradburd, G.S.}  P.L. Ralph, B.Moore, G.M. Coop. A Bayesian method for estimating genetic differentiation due to isolation %by geographic and ecological distance.  Society for the Study of Evolution. 2013. Ottawa, Canada.\

%\textbf{Bradburd, G.S.} Retrieving lost evolutionary history from an extinct darter species.  Ecology and Evolutionary Biology Senior Symposium, Yale University, 2008.

%\textbf{Bradburd, G.S.}, B.Moore, G.M. Coop, J.E. Foley. Molecular ecology of two tick-borne pathogens in California.  Center for Population Biology Departmental Seminar. 2011. UC Davis.\

