%COMPILE WITH XeLaTeX
\documentclass{res}
\setlength{\topmargin}{-1.0in}  
\setlength{\textheight}{9.8in}  
\setlength{\headsep}{0.2in}     
\setlength{\headheight}{10pt}   
\usepackage{fancyhdr}
\usepackage{setspace}
\usepackage{fontspec}
\setmainfont{Corbel}
\renewcommand{\headrulewidth}{0pt} 


\cfoot{}  
\pagestyle{fancy} 
\newenvironment{myindentpar}[1]%
    {\begin{list}{}%
             {\setlength{\leftmargin}{#1}}%
             \item[]}
    {\end{list}}
    
    
\begin{document} 
\thispagestyle{empty} 

\begin{resume}



\begin{center}
\begin{large}
\textbf{Gideon Bradburd\\
\vspace{2mm}
Curriculum Vitae}
\end{large}
\end{center}
%
\vspace{-0.5cm}
\noindent\makebox[\linewidth]{\rule{18cm}{0.4pt}}
%
\section{\textbf{Contact Information}}
Address: \hspace{43mm} 3348 Storer Hall\\
\hspace*{57mm} Evolution and Ecology\\
\hspace*{57mm} University of California, Davis\\
\hspace*{57mm} One Shields Avenue\\
\vspace{10pt}\hspace*{57mm} Davis, CA 95616\\
Phone: \hspace{46mm}w. 530-752-1112\\
\vspace{10pt}\hspace*{57mm} c. 203-687-5224\\
Email: \hspace{45mm} gbradburd@ucdavis.edu\\
\noindent\makebox[\linewidth]{\rule{18cm}{0.4pt}}

\section{\underline{EDUCATION}}
\textbf{University of California at Davis}, Davis, CA	(2009-present)\\
\vspace{3mm}
\hspace{6mm}{Ph.D. Candidate, Population Biology Graduate Group}\\
\textbf{Yale University}, New Haven, CT	(2004-2008)\\
\vspace{3mm}
\hspace{6mm}{Class of 2008, Ecology and Evolutionary Biology, B.S., GPA: 3.62}
%
\section{\underline{PUBLICATIONS}}
\textbf{Bradburd, G.S.}, P.L. Ralph, and G.M. Coop. Disentangling the effects of geographic and ecological isolation on genetic differentiation. Evolution 67(11):3258-3273.

Rejmanek, D, P. Freycon, \textbf{G.S. Bradburd}, J. Dinstell, and J. Foley.  Unique strains of \textit{Anaplasma phagocytophilum} segregate among diverse questing and non-questing \textit{Ixodes} tick species in the western United States.  Ticks and Tick-borne Diseases 4(6):482-487.

Rejmanek, D., \textbf{G.S. Bradburd}, and J. Foley.  2012.  Molecular characterization reveals	distinct genospecies of \textit{Anaplasma phagocytophilum} from diverse North American hosts.	Journal of Medical Microbiology. 61(2):204-212.

Near, T.J., C.M. Bossu, \textbf{G.S. Bradburd}, R.L. Carlson, R.C. Harrington, P.R. Hollingsworth	Jr., B.P. Keck, D.A. Etnier. 2011.  Phylogeny and temporal diversification of darters (\textit{Percidae: Etheostomatinae}).  Systematic Biology 60(5):565-595.

\section{\underline{FELLOWSHIPS AND AWARDS}}
\textbf{NSF Doctoral Dissertation Improvement Grant (DDIG)}, National Science Foundation, 2014-2015.\\
\textbf{Center for Population Biology Research Fellowship} Center for Population Biology, UC Davis, 2012-13.\\
\textbf{Center for Population Biology Research Fellowship} Center for Population Biology, UC Davis, 2011-12.\\
\textbf{NSF Graduate Research Fellow} National Science Foundation, 2010-'12\\ \vspace{0.3mm}
\textbf{Graduate Scholars Fellowship} in Population Biology, UC Davis, 2009-'10\\ \vspace{0.3mm}
\hspace{4.5mm}Awarded in recognition of outstanding academic record and promise of productive scholarship\\
\textbf{Integrative Biology Fellowship} UT Austin, 2009-'10 (declined)\\ \vspace{0.3mm}
\hspace{4.5mm}Awarded to top applicants in the UT system\\
\textbf{Dorothy Clark Lettice Fellowship} Kansas University, 2009-'10 (declined)\\  \vspace{0.3mm}
\hspace{4.5mm}Awarded in recognition of high academic ability\\
\textbf{Environmental Internship Award} Yale University, Summer 2007\\ 
\textbf{Richter Summer Travel Fellowship} Yale University, Summer 2007\\
\textbf{Mellon Forum Research Grant} Yale University, Summer 2007\\
\textbf{Environmental Internship Award} Yale University, Summer 2006\\

\section{\underline{INVITED TALKS}}

\textbf{Bradburd, G.S.}, P. Ralph, G. Coop. The Spatial Context of Genetic Admixture.  2014.  University of Montana.

\textbf{Bradburd, G.S.}, P. Ralph, G. Coop.  A Spatial Framework for Studying Genetic Differentiation.  2014.  University of Minnesota.

\textbf{Bradburd, G.S.} Disentangling the effects of geographic and ecological isolation on genetic differentiation.  2014.  Sonoma State University.

\textbf{Bradburd, G.S.} \textit{BEDASSLING} your data: background and functions of the R package \textit{BEDASSLE}.  RAD-seq Seminar. 2013. UT Austin.


\section{\underline{CONTRIBUTED PRESENTATIONS}}
\textbf{Society for Molecular Biology and Evolution}: 2014\\
\textbf{Society for the Study of Evolution}: 2013,2014\\
\textbf{CPB Dept. Seminar}:2011,2013\\
\textbf{E\&EB Senior Symposium, Yale University}: 2008\\

%\textbf{Bradburd, G.S.}, P.L. Ralph, G.M. Coop.   Disentangling the effects of geographic and ecological isolation on genetic differentiation.  Society for the Study of Evolution. 2013. Snowbird, Utah.\

%\textbf{Bradburd, G.S.}  P.L. Ralph, B.Moore, G.M. Coop. A Bayesian method for estimating genetic differentiation due to isolation %by geographic and ecological distance.  Society for the Study of Evolution. 2013. Ottawa, Canada.\

%\textbf{Bradburd, G.S.} Retrieving lost evolutionary history from an extinct darter species.  Ecology and Evolutionary Biology Senior Symposium, Yale University, 2008.

%\textbf{Bradburd, G.S.}, B.Moore, G.M. Coop, J.E. Foley. Molecular ecology of two tick-borne pathogens in California.  Center for Population Biology Departmental Seminar. 2011. UC Davis.\

\section{\underline{SERVICE}}
\textbf{Course Organizer}, The Bodega Bay Applied Phylogenetics Workshop (2012-present).\\
%
\textbf{Chair}, Center for Population Biology/Population Biology Grad Group Student Research Symposium (2013).\\
%
\textbf{Student Representative}, Population Biology Grad Group Steering Committee(2013-present).\\
%
\textbf{Graduate Student Mentor} (2009-present). \\
%
\textbf{Chair}, UC Davis Non-Model Genomics Workshop Committee (2010).
%
\section{\underline{REVIEWER}}
\textbf{Evolution}, 
\textbf{Methods in Ecology and Evolution}, 
\textbf{Diversity and Distributions},
\textbf{Integrative and Comparative Biology}

\section{\underline{Society Memberships}}
\textbf{Society for the Study of Evolution},

\section{\underline{WORK EXPERIENCE}}
\textbf{Research Assistant, Prof. Thomas Near, Yale University} 2008-'09 \\ \vspace{0.3mm}	%(40 hrs/wk)
\hspace{4.5mm}Carried out research on phylogeography of North American fishes\\
\textbf{Technician, Yale Peabody Museum of Natural History} 2007-'08 \\ \vspace{0.3mm}	%(8 hrs/wk)
\hspace{4.5mm}Identified and sorted vertebrate specimens\\ 
\textbf{Lab Technician, Marine Biology Lab, Prof. Leo Buss} Summer 2005 \\ \vspace{0.3mm}	%(25 hrs/wk)
\hspace{4.5mm}Maintained populations of \textit{Hydractinia echinata} and \textit{Nematostella vectensis} 

\section{\underline{FIELDWORK}}
\textbf{Northern California}, 2010-2012\\\vspace{0.3mm}
\hspace{4.5mm}Sampled tissue and ectoparasites from lizards and small rodents.\\
\textbf{American Southeast and Midwest} Summer, Fall 2008\\ \vspace{0.3mm}
\hspace{4.5mm}Sampled streams for Cyprinids, Centrarchids, and Percids throughout the Mississippi Drainage.\\
\textbf{Costa Rica} Summer 2007\\ \vspace{0.3mm}
\hspace{4.5mm}Collected reptiles and amphibians in the Costa Rican rainforest for the Yale Peabody Museum.\\ \vspace{0.3mm}
\hspace{4.5mm}Developed skills in collecting and preserving herpetological specimens.\\\vspace{0.3mm}
\hspace{4.5mm}Monitored status of last known population of the Variable Harlequin Toad (\textit{Atelopus varius}).\\
\textbf{Suriname} Summer 2006\\ \vspace{0.3mm}
\hspace{4.5mm}Collected birds in coastal and southern Suriname for the Yale Peabody Museum.\\ \vspace{0.3mm}
\hspace{4.5mm}Developed skills in collecting (mist-netting, shooting) and preserving ornithological specimens.\\ \vspace{0.3mm} 
\hspace{4.5mm}(anatomicals, skeletons, skins).\\ 





\section{\underline{LANGUAGES}}
\textbf{Spanish}: proficient speaking and reading\\
\textbf{Sranan Tongo (Taki-taki)}: fair, but only in the context of collecting specimens in the field



\end{resume}
\end{document}

%\textbf{Relevant Coursework}
%\begin{myindentpar}{6mm}
%{UC Davis: Core Principles in Population Biology, Mathematical Methods in Population Biology, Macroevolution, Principles of Biological Data %Analysis}\vspace{3mm}

%Yale: General Chemistry (w/ lab), General Physics, General Ecology, Evolutionary Biology, Diversity of Life, Molecular Systematics (intensive lab), %Genetics, Conservation Genetics, Phylogenetic Methods and Approaches, Biology of Terrestrial Arthropods (w/ lab), Ornithology (w/ lab), %Introductory Geoscience, Petrology and Mineralogy, Natural Resources and Sustainability
%\end{myindentpar}


%\section{\underline{VOLUNTEER AND LEADERSHIP EXPERIENCE}}
%\textbf{Yale Freshman Outdoor Orientation Trips (FOOT) Leader} 2005-'07\\ \vspace{0.3mm}
%\hspace{4.5mm}Led 4-6 day trips in the wilderness with 8-10 incoming Yale freshmen\\ \vspace{0.3mm}
%\hspace{4.5mm}Trained and certified in Wilderness First Aid and CPR\\
%\textbf{FOOT Public Relations Core Head} 2006-'07\\ \vspace{0.3mm}
%\hspace{4.5mm}Organized all public relations initiatives for the FOOT program, including sponsorship bids, new\\ \vspace{0.3mm} 
%\hspace{4.5mm}leader recruitment, and advertising to incoming freshmen\\ 
%\textbf{Volunteer Income Tax Assistance Site Coordinator, New Haven} 2005-'07\\ \vspace{0.3mm}
%\hspace{4.5mm}Coordinated an office and trained volunteers to administer Earned Income Tax Credit to local\\ \vspace{0.3mm}
%\hspace{4.5mm}individuals and families\\ 
%\textbf{Local Director of Client Services, National Student Partnerships, New Haven} 2004-'06\\ \vspace{0.3mm}
%\hspace{4.5mm}Worked with socioeconomically disadvantaged clients to navigate social bureaucracy and find\\ \vspace{0.3mm}
%\hspace{4.5mm}employment, affordable housing, and public benefits\\ \vspace{0.3mm}
%\textbf{Community Health Educator, New Haven} 2005-'06\\ \vspace{0.3mm}
%\hspace{4.5mm}Taught in local schools to provide health education, specifically in communication skills with\\ \vspace{0.3mm}
%\hspace{4.5mm}reference to substance and relationship abuse, nutrition, and sexual health\\
%\textbf{DEMOS Classroom Volunteer, Lincoln-Bassett Elementary School, New Haven} 2004-'05\\ \vspace{0.3mm}
%\hspace{4.5mm}Demonstrated basic scientific principles in a fun and accessible way\\ 