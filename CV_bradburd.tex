%COMPILE WITH LuaLaTeX
\documentclass{article}
\usepackage[charter]{mathdesign}
%\usepackage{times}
\usepackage{xcolor}
\usepackage{hyperref}
\usepackage{array}
\hypersetup{colorlinks,urlcolor=blue}
\usepackage[margin=1in]{geometry}
\usepackage{setspace}
\usepackage{fontspec}
\setmainfont[%
  Path = /Library/Fonts/,
  UprightFont = Corbeln,
  BoldFont = Corbelb,
  ItalicFont = Corbeli,
  Extension = .ttf
]{Corbel}
\usepackage{breakurl}
\newcolumntype{R}{>{\raggedleft\arraybackslash}p{0.295\textwidth}}
\setlength{\parindent}{0in}
\usepackage[bottom]{footmisc}
\usepackage{tabularx}
\usepackage{array}
\begin{document}
\pagenumbering{gobble}
%
\begin{huge}
\bf{Gideon Bradburd}\
\end{huge}
\vspace{-0.25cm}
\\
%
\vspace{-0.7cm}
\rule{470pt}{0.4pt}
%\noindent\makebox[\linewidth]{\rule{18cm}{0.4pt}}
\vspace{0.3cm}
%
\\
153 Giltner Hall \hfill email: \href{mailto:bradburd@msu.edu}{bradburd@msu.edu}\\
Department of Integrative Biology, MSU \hfill website: \href{http://genescape.org}{genescape.org}\\
%\hfill phone: 203-687-5224\\
East Lansing, MI, 48824 \hfill \\
%\vspace{-0.1cm}
\section*{EDUCATION \& APPOINTMENTS}
\vspace{-0.6cm}
\rule{470pt}{0.4pt}
%
\begin{tabular}{@{}p{0.1\textwidth}p{0.50\textwidth}R@{}}
\\
\bf{Asst. Prof} & \it{Michigan State University} & Aug. 2016 - present\\
 & \hspace{0.5cm}Department of Integrative Biology & \\
 & \hspace{0.5cm}Ecology, Evolutionary Biology, and Behavior Group& \\
%
%\hfill\\
%
\bf{Postdoc} & \it{University of California, Berkeley} & Sept. 2015 - Jul. 2016\\
 & \hspace{0.5cm}Museum of Vertebrate Zoology & \\
 & \hspace{0.5cm}Department of Env.\ Sci.\ Pol.\ and Mgmt& \\
 & \hspace{0.5cm}Advisors: Michael Nachman, Bree Rosenblum & \\ 
%
%\hfill\\
%
\bf{PhD} & \it{University of California, Davis}  & Sept. 2009 - Aug. 2015\\
 & \hspace{0.5cm}Population Biology Graduate Group & \\
 & \hspace{0.5cm}Department of Ecology and Evolutionary Biology & \\
 & \hspace{0.5cm}Advisors: Graham Coop, Brad Shaffer & \\ 
%
%\hfill\\
%
\textbf{BS} & \it{Yale University} & Sept. 2004 - May 2008 \\
& \hspace{0.5cm}Ecology and Evolutionary Biology & \\
& \hspace{0.5cm}Honors Thesis Advisor: Tom Near & \\
\end{tabular}
%
\section*{HONORS, AWARDS, \& FELLOWSHIPS}
\vspace{-0.6cm}
\rule{470pt}{0.4pt}
%
\begin{tabular}{>{\everypar{\hangindent1cm}}p{0.85\textwidth}p{0.1\textwidth}}
\hfill\\
\textbf{Fitch Award Finalist}, Society for Molecular Biology and Evolution (\$2000) & \hfill 2015\\
\textbf{Hamilton Award Finalist}, Society for the Study of Evolution (\$500) & \hfill 2015\\
\textbf{XSEDE SuperComputing Resource Allocation}, XSEDE (50,000 SUs) & \hfill 2014-05\\
\textbf{Center for Population Biology Research Fellowship} CPB, UC Davis (\$4,000) & \hfill 2011-13\\
%\textbf{Center for Population Biology Research Fellowship} CPB, UC Davis (\$2,000) & \hfill 2011-12\\
\textbf{NSF Graduate Research Fellow} National Science Foundation (\$90,000) & \hfill 2010-12\\
\textbf{Graduate Scholars Fellowship} in Population Biology, UC Davis (\$30,000) & \hfill  2009-10\\
%\hspace{4.5mm}Awarded in recognition of outstanding academic record and promise of productive scholarship\\
\textbf{Environmental Internship Award} Yale University (\$6,000) & \hfill 2006-7\\ 
\textbf{Richter Summer Travel Fellowship} Yale University (\$1,000)  & \hfill 2007\\
\textbf{Mellon Forum Research Grant} Yale University (\$500) & \hfill 2007\\
%\textbf{Environmental Internship Award} Yale University (\$2,000) & \hfill 2006\\
\end{tabular}
%
\section*{FUNDING}
\vspace{-0.6cm}
\rule{470pt}{0.4pt}
\begin{tabular}{>{\everypar{\hangindent1cm}}p{0.8\textwidth}p{0.15\textwidth}}
%
\textbf{NSF Dimensions of Biodiversity} (2017-2021) & \hfill \textbf{\$1,846,134} \\
\hspace{4.5mm}Diversification of sensory systems in novel habitat: enhanced vision or compensation in other modalities?\\
\hspace{4.5mm}(co-PIs: J.W. Boughman, H. Hofmann, J. Keagy, D. Stenkamp)\\
%
\textbf{Beacon - NSF Center for the Study of Evolution in Action} (2016-present) & \hfill \textbf{\$127,579}\\
\hspace{4.5mm}Evolution of sensory systems in response to loss of visual information\\
\hspace{4.5mm}(co-PIs: J.W. Boughman, H. Hofmann, J. Keagy, D. Stenkamp)\\
%
\textbf{NSF Doctoral Dissertation Improvement Grant (DDIG)} (2014-15) & \hfill \textbf{\$13,918}\\
\hspace{4.5mm}The effect of intraspecific host variation on the structure of parasite populations\\
\hspace{4.5mm}(co-PIs: Graham Coop)\\
\end{tabular}
%
\section*{PUBLICATIONS}
\vspace{-0.6cm}
\rule{470pt}{0.4pt}
%
\begin{tabular}{>{\everypar{\hangindent1cm}}p{0.95\textwidth}p{0.05\textwidth}}
\hfill\\
\underline{Preprints and In Review} \hfill\\
\vspace{-0.1cm}
\textbf{Bradburd, G.S.}, P.L. Ralph, and G.M. Coop. (2017). 
Inferring continuous and discrete population structure across space \underline{bioRxiv} 
\href{https://doi.org/10.1101/189688}{https://doi.org/10.1101/189688}. \\ 
%
\vspace{-0.1cm}
Shaffer, H.B., McCartney-Melstad, E., Ralph, P.L., \textbf{Bradburd, G.S.}, Lundgren, E., Vu, J., Hagerty, B., Sandmeier, F., Weitzman, C., Tracy, C.R.
(2017). Desert Tortoises in the Genomic Age: Population Genetics and the Landscape \underline{bioRxiv}
\href{https://doi.org/10.1101/195743}{https://doi.org/10.1101/195743}.\\
%
\vspace{0.2cm}
%
\underline{Published and Accepted Papers} \hfill\\
\vspace{-0.1cm}
Weber, J.N., \textbf{G.S. Bradburd}, Y.E. Stuart, W.E. Stutz, D.I. Bolnick. (2017).
Partitioning the effects of isolation by distance, environment, and physical barriers 
on genomic divergence between parapatric threespine stickleback.
\underline{Evolution} 71(2):342-56. \\
%
\vspace{-0.1cm}
Hoban, S.*, J. L. Kelley*, K. E. Lotterhos*, M. F. Antolin, \textbf{G.S. Bradburd}, D. B. Lowry, M. L. Poss, L. K. Reed, A. Storfer, M. C. Whitlock. (2016). Finding the genomic basis of local adaptation in non-model organisms: pitfalls, practical solutions, and future directions. \underline{American Naturalist} 188(4). (*denotes equal authorship)\\
%
\vspace{-0.1cm}
\textbf{Bradburd G.S.}, P.L. Ralph, G.M. Coop. (2016). A Spatial Framework for Understanding Population Structure and Admixture. \underline{PLoS Genetics} 12(1): e1005703.\\
%
\vspace{-0.1cm}
Hammock, B.G., S. Lesmeister, I. Flores, \textbf{G.S. Bradburd}, F.H. Hammock, S.J. Teh. (2016). Low food availability narrows the tolerance of the copepod \textit{Eurytemora affinis} to salinity, but not to temperature. \underline{Estuaries and Coasts}.  39(1):189-200.\\
%
\vspace{-0.1cm}
Agrawal, A.A., A.P. Hastings, \textbf{G.S. Bradburd}, E.C. Woods, T. Z{\"u}st, J. Harvey, T. Bukovinszky. (2015).
Evolution of plant growth and defense in a continental introduction. \underline{American Naturalist} 186(1): E1-E15.\\
%
\vspace{-0.1cm}
Wang, I.J.* and \textbf{G.S. Bradburd}.* (2014). Isolation by Environment. \underline{Molecular Ecology} 23: 5649-5662.  & \hfill\\
\hspace{4.5mm} (*denotes equal authorship)&\\
%
\vspace{-0.1cm}
%
\textbf{Bradburd, G.S.}, P.L. Ralph, and G.M. Coop. (2013). Disentangling the effects of geographic and ecological isolation on genetic differentiation. \underline{Evolution} 67: 3258-3273. & \hfill \\
%
\vspace{-0.1cm}
%
Rejmanek, D, P. Freycon, \textbf{G.S. Bradburd}, J. Dinstell, and J. Foley.  (2013). Unique strains of \textit{Anaplasma phagocytophilum} segregate among diverse questing and non-questing \textit{Ixodes} tick species in the western United States.  \underline{Ticks and Tick-borne Diseases} 4: 482-487. & \hfill\\
%
\vspace{-0.1cm}
%
Rejmanek, D., \textbf{G.S. Bradburd}, and J. Foley.  (2012). Molecular characterization reveals distinct genospecies of \textit{Anaplasma phagocytophilum} from diverse North American hosts.	\underline{Journal of Medical Microbiology} 61: 204-212. & \hfill \\
%
\vspace{-0.1cm}
%
Near, T.J., C.M. Bossu, \textbf{G.S. Bradburd}, R.L. Carlson, R.C. Harrington, P.R. Hollingsworth Jr., B.P. Keck, D.A. Etnier.  (2011). Phylogeny and temporal diversification of darters (\textit{Percidae: Etheostomatinae}).  \underline{Systematic Biology} 60: 565-595& \hfill\\
%
\end{tabular}
%
%\newpage
\section*{SOFTWARE}
\vspace{-0.6cm}
\rule{470pt}{0.4pt}
\begin{tabular}{>{\everypar{\hangindent1cm}}p{0.95\textwidth}p{0.05\textwidth}}
\hfill\\
%
Bradburd, G.S. 2017. conStruct. A genetic clustering method that accounts for isolation by distance. R package version 0.0.0.9000.
\href{https://github.com/gbradburd/conStruct/code/conStruct}{https://github.com/gbradburd/conStruct/code/conStruct}\\
%
Bradburd, G.S. 2016. SpaceMix. A program for estimating geogenetic maps from allele frequency data. R package version 0.13.
\href{https://github.com/gbradburd/SpaceMix}{https://github.com/gbradburd/SpaceMix}\\
%
\vspace{-0.1cm}
%
Bradburd, G.S. 2013. BEDASSLE. Bayesian Estimation of Differentiation in Alleles by Spatial Structure and Local Ecology. 
R package version 1.5. \href{https://cran.r-project.org/web/packages/BEDASSLE}{https://cran.r-project.org/web/packages/BEDASSLE}\\
%
\end{tabular}
%
\section*{TEACHING EXPERIENCE}
\vspace{-0.6cm}
\rule{470pt}{0.4pt}
%
\begin{tabular}{>{\everypar{\hangindent1cm}}p{0.60\textwidth}p{0.225\textwidth}p{0.1\textwidth}}
\hfill\\
Intro Quantitative Methods in Biology (IBio 851) & Instructor & \hfill 2017 \\
Evolution Discussion Group (IBio 895) & Instructor & \hfill 2017 \\
UCLA/La Kretz Center Workshop in Conservation Genomics & Instructor & \hfill 2014, 16 \\
EVE100 Introduction to Evolution & Teaching Assistant & \hfill 2015	\\
EVE103 Macroevolution, Speciation, and Phylogeny (Lecture, Lab) & Teaching Assistant & \hfill 2014 \\
Herpetology (Field, Lab, Lecture) & Teaching Assistant & \hfill 2012\\
Bodega Bay Applied Phylogenetics Workshop & Teaching Assistant & \hfill 2012-13\\
\end{tabular}
%
%
\section*{INVITED TALKS}
\vspace{-0.6cm}
\rule{470pt}{0.4pt}
%
\begin{tabular}{>{\everypar{\hangindent1cm}}p{0.4\textwidth}p{0.4\textwidth}p{0.1\textwidth}}
\hfill\\
University of Chicago & Dept. of Ecology and Evolution & \hfill 2017 \\
%
Michigan State University & Kellogg Biological Station & \hfill 2017 \\
%
University of Michigan & Center for Statistical Genetics & \hfill 2016 \\
%
University of Wyoming & Dept. of Botany & \hfill 2016 \\
%
University of Colorado Boulder & Dept. Ecology and Evolutionary Biology & \hfill 2016 \\
%
University of San Francisco & Dept. of Biology & \hfill 2016 \\
%
University of California Berkeley & Museum of Vertebrate Zoology & \hfill 2015 \\
%
Michigan State University & Ecology, Evolutionary Biology and Behavior & \hfill 2015 \\
%
Stanford University & Dept. Genetics and Biology & \hfill 2015 \\
%
University of Southern California & Molecular and Computational Biology & \hfill 2014 \\
%
University of Montana & Division of Biological Sciences & \hfill 2014 \\
%
University of Minnesota & Dept. of Plant Biology & \hfill 2014 \\
%
Sonoma State University & Biology Dept. & \hfill 2014 \\
%
University of Texas, Austin & Dept. Integrative Biology & \hfill 2013 \\
%
\end{tabular}
%
\section*{CONTRIBUTED PRESENTATIONS}
\vspace{-0.6cm}
\rule{470pt}{0.4pt}
\begin{tabular}{>{\everypar{\hangindent1cm}}p{0.85\textwidth}p{0.1\textwidth}}
\hfill\\
Midwestern Population Genetics Meeting & \hfill 2016\\
American Society of Naturalists & \hfill 2016\\
Society for Molecular Biology and Evolution & \hfill 2015\\
%International Association of Landscape Ecology & \hfill 2015\\
Society for the Study of Evolution & \hfill 2012-13,15\\
Bay Area Population Genomics (BAPG) & \hfill 2014\\
%CPB Dept. Seminar & \hfill 2011,13\\
E\&EB Senior Symposium, Yale University & \hfill 2008\\
\end{tabular}

\section*{SERVICE \& LEADERSHIP}
\vspace{-0.6cm}
\rule{470pt}{0.4pt}
\begin{tabular}{>{\everypar{\hangindent1cm}}p{0.8\textwidth}p{0.15\textwidth}}
\hfill\\
Organizing Committee, Midwest Population Genetics Conference & \hfill 2016-present\\
%
Curriculum Planning Committee (Intro to Evolution - IBio 445) & \hfill 2016-17\\
%
Student Representative, Population Biology Grad Group Steering Committee & \hfill 2013-14\\
%
Chair of Student Research Symposium, Center for Population Biology/Pop Bio Grad Group & \hfill 2013\\
%
Course Organizer, The Bodega Bay Applied Phylogenetics Workshop & \hfill 2012-13\\
%
Chair, UC Davis Non-Model Genomics Workshop Committee & \hfill 2010\\
%
Graduate Student Mentor & \hfill 2009-2015\\
%
\end{tabular}
%
\section*{REVIEWER}
\vspace{-0.6cm}
\rule{470pt}{0.4pt}
\\\\
Evolution;
Genetics;
Heredity;
American Journal of Human Genetics;
Molecular Ecology;
Molecular Ecology Resources;
Methods in Ecology and Evolution;
New Phytologist;
Bioinformatics;
Diversity and Distributions;
Integrative and Comparative Biology;
Nature - Scientific Reports;
Genes, Genomes, Genetics (G3);
NSERC


\section*{SOCIETY MEMBERSHIPS}
\vspace{-0.6cm}
\rule{470pt}{0.4pt}
\\\\
Society for the Study of Evolution (SSE); Society for Molecular Biology and Evolution (SMBE); American Society of Naturalists (ASN)

\end{document}

\section*{PREVIOUS APPOINTMENTS}
\vspace{-0.6cm}
\rule{470pt}{0.4pt}
\begin{tabular}{>{\everypar{\hangindent1cm}}p{0.8\textwidth}p{0.15\textwidth}}
\hfill\\
\textbf{Research Assistant, Prof. Thomas Near, Yale University} & \hfill 2008-09 \\ 
\hspace{4.5mm}Carried out research on phylogeography of North American fishes & \\
%
\textbf{Technician, Yale Peabody Museum of Natural History} & \hfill 2007-08 \\ 
\hspace{4.5mm}Identified and sorted vertebrate specimens&\\ 
%
\textbf{Lab Technician, Marine Biology Lab, Prof. Leo Buss} & \hfill 2005 \\ 
\hspace{4.5mm}Maintained populations of \textit{Hydractinia echinata} and \textit{Nematostella vectensis}&
\end{tabular}



\section*{FIELDWORK}
\vspace{-0.6cm}
\rule{470pt}{0.4pt}
\begin{tabular}{>{\everypar{\hangindent1cm}}p{0.8\textwidth}p{0.15\textwidth}}
\hfill\\
\textbf{Abaco Islands (Bahamas)} & \hfill 2013\\
\hspace{4.5mm}Arthropod surveys and sampling; Anole sampling.\\
%
\textbf{Northern California} & \hfill 2010-2012\\
\hspace{4.5mm}Sampled tissue and ectoparasites from lizards and small rodents.\\
%
\textbf{American Southeast and Midwest} & \hfill 2008\\ 
\hspace{4.5mm}Sampled freshwater fish throughout the Mississippi Drainage.\\
%
\textbf{Costa Rica}  & \hfill  2007\\ 
\hspace{4.5mm}Collected reptiles and amphibians in the rainforest for the Yale Peabody Museum.\\ 
\hspace{4.5mm}Developed skills in collecting and preserving herpetological specimens.\\
\hspace{4.5mm}Monitored status of the Variable Harlequin Toad (\textit{Atelopus varius}).\\
%
\textbf{Suriname}  & \hfill  2006\\
\hspace{4.5mm}Collected birds in coastal and southern Suriname for the Yale Peabody Museum.\\ 
\hspace{4.5mm}Developed skills in collecting (mist-netting, shooting) and preserving ornithological specimens (anatomicals, skeletons, skins).\\ 
\end{tabular}





\section{\underline{LANGUAGES}}
\textbf{Spanish}: proficient speaking and reading\\
\textbf{Sranan Tongo (Taki-taki)}: fair, but only in the context of collecting specimens in the field

%\end{resume}


%GRAVEYARD
\textbf{Bradburd, G.S.}, P.L. Ralph, and G.M. Coop. Disentangling the effects of geographic and ecological isolation on genetic differentiation. Evolution 67(11):3258-3273.
%
Rejmanek, D, P. Freycon, \textbf{G.S. Bradburd}, J. Dinstell, and J. Foley.  Unique strains of \textit{Anaplasma phagocytophilum} segregate among diverse questing and non-questing \textit{Ixodes} tick species in the western United States.  Ticks and Tick-borne Diseases 4(6):482-487.
%
Rejmanek, D., \textbf{G.S. Bradburd}, and J. Foley.  2012.  Molecular characterization reveals	distinct genospecies of \textit{Anaplasma phagocytophilum} from diverse North American hosts.	Journal of Medical Microbiology. 61(2):204-212.
%
Near, T.J., C.M. Bossu, \textbf{G.S. Bradburd}, R.L. Carlson, R.C. Harrington, P.R. Hollingsworth	Jr., B.P. Keck, D.A. Etnier. 2011.  Phylogeny and temporal diversification of darters (\textit{Percidae: Etheostomatinae}).  Systematic Biology 60(5):565-595.


%\cfoot{}  
%\pagestyle{fancy} 
%\newenvironment{myindentpar}[1]%
%   {\begin{list}{}%
%             {\setlength{\leftmargin}{#1}}%
%             \item[]}
%    {\end{list}}

%\textbf{Relevant Coursework}
%\begin{myindentpar}{6mm}
%{UC Davis: Core Principles in Population Biology, Mathematical Methods in Population Biology, Macroevolution, Principles of Biological Data %Analysis}\vspace{3mm}

%Yale: General Chemistry (w/ lab), General Physics, General Ecology, Evolutionary Biology, Diversity of Life, Molecular Systematics (intensive lab), %Genetics, Conservation Genetics, Phylogenetic Methods and Approaches, Biology of Terrestrial Arthropods (w/ lab), Ornithology (w/ lab), %Introductory Geoscience, Petrology and Mineralogy, Natural Resources and Sustainability
%\end{myindentpar}

\textbf{Integrative Biology Fellowship} UT Austin, 2009-'10 (declined)\\ \vspace{0.3mm}
\hspace{4.5mm}Awarded to top applicants in the UT system\\
\textbf{Dorothy Clark Lettice Fellowship} Kansas University, 2009-'10 (declined)\\  \vspace{0.3mm}
\hspace{4.5mm}Awarded in recognition of high academic ability\\

%\section{\underline{VOLUNTEER AND LEADERSHIP EXPERIENCE}}
%\textbf{Yale Freshman Outdoor Orientation Trips (FOOT) Leader} 2005-'07\\ \vspace{0.3mm}
%\hspace{4.5mm}Led 4-6 day trips in the wilderness with 8-10 incoming Yale freshmen\\ \vspace{0.3mm}
%\hspace{4.5mm}Trained and certified in Wilderness First Aid and CPR\\
%\textbf{FOOT Public Relations Core Head} 2006-'07\\ \vspace{0.3mm}
%\hspace{4.5mm}Organized all public relations initiatives for the FOOT program, including sponsorship bids, new\\ \vspace{0.3mm} 
%\hspace{4.5mm}leader recruitment, and advertising to incoming freshmen\\ 
%\textbf{Volunteer Income Tax Assistance Site Coordinator, New Haven} 2005-'07\\ \vspace{0.3mm}
%\hspace{4.5mm}Coordinated an office and trained volunteers to administer Earned Income Tax Credit to local\\ \vspace{0.3mm}
%\hspace{4.5mm}individuals and families\\ 
%\textbf{Local Director of Client Services, National Student Partnerships, New Haven} 2004-'06\\ \vspace{0.3mm}
%\hspace{4.5mm}Worked with socioeconomically disadvantaged clients to navigate social bureaucracy and find\\ \vspace{0.3mm}
%\hspace{4.5mm}employment, affordable housing, and public benefits\\ \vspace{0.3mm}
%\textbf{Community Health Educator, New Haven} 2005-'06\\ \vspace{0.3mm}
%\hspace{4.5mm}Taught in local schools to provide health education, specifically in communication skills with\\ \vspace{0.3mm}
%\hspace{4.5mm}reference to substance and relationship abuse, nutrition, and sexual health\\
%\textbf{DEMOS Classroom Volunteer, Lincoln-Bassett Elementary School, New Haven} 2004-'05\\ \vspace{0.3mm}
%\hspace{4.5mm}Demonstrated basic scientific principles in a fun and accessible way\\ 
%\textbf{Bradburd, G.S.}, P.L. Ralph, G.M. Coop.   Disentangling the effects of geographic and ecological isolation on genetic differentiation.  Society for the Study of Evolution. 2013. Snowbird, Utah.\

%\textbf{Bradburd, G.S.}  P.L. Ralph, B.Moore, G.M. Coop. A Bayesian method for estimating genetic differentiation due to isolation %by geographic and ecological distance.  Society for the Study of Evolution. 2013. Ottawa, Canada.\

%\textbf{Bradburd, G.S.} Retrieving lost evolutionary history from an extinct darter species.  Ecology and Evolutionary Biology Senior Symposium, Yale University, 2008.

%\textbf{Bradburd, G.S.}, B.Moore, G.M. Coop, J.E. Foley. Molecular ecology of two tick-borne pathogens in California.  Center for Population Biology Departmental Seminar. 2011. UC Davis.\

%\section*{INVITED TALKS}
%\vspace{-0.6cm}
%\rule{470pt}{0.4pt}
%%
%\begin{tabular}{>{\everypar{\hangindent1cm}}p{0.85\textwidth}p{0.1\textwidth}}
%\hfill\\
%%
%\textbf{Bradburd, G.S.} Patterns of Spatial Genetic Variation. & \hfill 2015 \\
%\hspace{1cm} Michigan State University.\\
%%
%\textbf{Bradburd, G.S.}, P. Ralph, G. Coop. The Geography of Genetic Structure and Admixture. & \hfill 2015 \\
%\hspace{1cm} Stanford University.\\
%%
%\textbf{Bradburd, G.S.}, P. Ralph, G. Coop. A new approach to the geography of genetic structure. & \hfill 2014 \\
%\hspace{1cm} University of Southern California.\\
%%
%\textbf{Bradburd, G.S.}, P. Ralph, G. Coop. The spatial context of genetic admixture. & \hfill 2014 \\
%\hspace{1cm} University of Montana.\\
%%
%\textbf{Bradburd, G.S.}, P. Ralph, G. Coop.  A spatial framework for studying genetic differentiation.  & \hfill 2014 \\
%\hspace{1cm} University of Minnesota.\\
%%
%\textbf{Bradburd, G.S.} The effects of geographic and ecological isolation on genetic differentiation. & \hfill 2014 \\
%\hspace{1cm} Sonoma State University. \\
%%
%\textbf{Bradburd, G.S.} \textit{BEDASSLING} your data: background and functions of the R package \textit{BEDASSLE}.  & \hfill 2013 \\
%\hspace{1cm} University of Texas, Austin\\
%\end{tabular}