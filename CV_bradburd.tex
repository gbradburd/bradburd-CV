%COMPILE WITH LuaLaTeX
\documentclass{article}
\usepackage[charter]{mathdesign}
%\usepackage{times}
\usepackage{xcolor}
\usepackage{longtable}
\usepackage{hyperref}
\usepackage{array}
\hypersetup{colorlinks,urlcolor=blue}
\usepackage[margin=1in]{geometry}
\usepackage{setspace}
\usepackage{fontspec}
%\usepackage{etaremune}
\usepackage{fancyhdr}
\setmainfont[%
  Path = /Library/Fonts/,
  UprightFont = Corbel,
  BoldFont = Corbelb,
  ItalicFont = Corbeli,
  Extension = .ttf
]{Corbel}
\usepackage{breakurl}
\newcolumntype{R}{>{\raggedleft\arraybackslash}p{0.295\textwidth}}
\setlength{\parindent}{0in}
\usepackage[bottom]{footmisc}
\usepackage{tabularx}
\usepackage{array}
\pagestyle{fancy}
\renewcommand{\headrule}{}
\fancyhead[R]{\textit{Bradburd - Curriculum Vitae}}
%
%
\begin{document}\thispagestyle{empty}
\pagenumbering{gobble}
%
\begin{huge}
\bf{Gideon Bradburd}\
\end{huge}
\vspace{-0.25cm}
\\
%
\vspace{-0.7cm}
\rule{470pt}{0.4pt}
%\noindent\makebox[\linewidth]{\rule{18cm}{0.4pt}}
\vspace{0.3cm}
%
\\
153 Giltner Hall \hfill email: \href{mailto:bradburd@msu.edu}{bradburd@msu.edu}\\
Department of Integrative Biology, MSU \hfill website: \href{http://genescape.org}{genescape.org}\\
%\hfill phone: 203-687-5224\\
East Lansing, MI, 48824 \hfill software: \href{https://github.com/gbradburd}{github.com/gbradburd}\\
%\vspace{-0.1cm}
\section*{EDUCATION \& APPOINTMENTS}
\vspace{-0.6cm}
\rule{470pt}{0.4pt}
%
\begin{tabular}{p{0.25\textwidth} p{0.5\textwidth} p{0.175\textwidth}} 
%L{0.25\textwidth} L{0.5\textwidth} c
%p{0.22\textwidth}p{0.49\textwidth}p{0.2\textwidth}
\\
\bf{Asst. Prof} & \it{Michigan State University} & \hfill 2017 - present\\
 & \hspace{0.5cm}Department of Integrative Biology & \\
 & \hspace{0.5cm}Ecology, Evolution, and Behavior Program& \\
%
%
\bf{Fixed-term Asst. Prof} & \it{Michigan State University} & \hfill 2016 - 2017\\
 & \hspace{0.5cm}Department of Integrative Biology & \\
 & \hspace{0.5cm}Ecology, Evolution, and Behavior Program& \\
%
%
\bf{Postdoc} & \it{University of California, Berkeley} & \hfill 2015 - 2016\\
 & \hspace{0.5cm}Museum of Vertebrate Zoology & \\
 & \hspace{0.5cm}Department of Env.\ Sci.\ Pol.\ and Mgmt& \\
 & \hspace{0.5cm}Advisors: Michael Nachman, Bree Rosenblum & \\ 
%
%
\bf{PhD} & \it{University of California, Davis}  & \hfill 2009 - 2015\\
 & \hspace{0.5cm}Population Biology Graduate Group & \\
 & \hspace{0.5cm}Department of Ecology and Evolutionary Biology & \\
 & \hspace{0.5cm}Advisors: Graham Coop, Brad Shaffer & \\ 
%
%
\textbf{BS} & \it{Yale University} & \hfill 2004 - 2008 \\
& \hspace{0.5cm}Ecology and Evolutionary Biology & \\
& \hspace{0.5cm}Honors Thesis Advisor: Tom Near & \\
\end{tabular}
%
\vspace{0.1cm}
\section*{FUNDING}
\vspace{-0.6cm}
\rule{470pt}{0.4pt}
\begin{tabular}{>{\everypar{\hangindent1cm}}p{0.8\textwidth}p{0.15\textwidth}}
\hfill\\
%
\textbf{NIH Early Stage Investigator MIRA (R35)} (2020-2025) & \hfill \textbf{\$1,849,678} \\
\hspace{4.5mm} Incorporating geography into statistical methods for analysis of population genomic DNA\\
\hspace{4.5mm}(PI: G. Bradburd)\\ \vspace{-0.1cm}
%
\textbf{NSF Bridging Ecology \& Evolution} (2020-2023) & \hfill \textbf{\$920,700} \\
\hspace{4.5mm} Testing eco-evolutionary effects of genetic drift and gene flow in stressful environments\\
\hspace{4.5mm}(PI: S. Fitzpatrick, co-PIs: B. Rothermel, G. Bradburd)\\ \vspace{-0.1cm}
%
\textbf{NSF Dimensions of Biodiversity} (2017-2021) & \hfill \textbf{\$1,118,520} \\
\hspace{4.5mm}Diversification of sensory systems in novel habitat: enhanced vision or compensation in other modalities?\\
\hspace{4.5mm}(PI: J. Boughman, co-PIs: G. Bradburd, H. Hofmann, J. Keagy, D. Stenkamp)\\ \vspace{-0.1cm}
%
\textbf{Beacon - NSF Center for the Study of Evolution in Action} (2016-present) & \hfill \textbf{\$127,579}\\
\hspace{4.5mm}Evolution of sensory systems in response to loss of visual information\\
\hspace{4.5mm}(PI: J. Boughman, co-PIs: G. Bradburd, H. Hofmann, J. Keagy, D. Stenkamp)\\ \vspace{-0.1cm}
%
\textbf{NSF Doctoral Dissertation Improvement Grant (DDIG)} (2014-15) & \hfill \textbf{\$13,918}\\
\hspace{4.5mm}The effect of intraspecific host variation on the structure of parasite populations\\
\hspace{4.5mm}(PI: G. Bradburd, co-PI: G. Coop)\\ \vspace{-0.1cm}
\end{tabular}
%
\newpage
\section*{PUBLICATIONS}
\vspace{-0.6cm}
\rule{470pt}{0.4pt}
%
\vspace{-0.9cm}
\newcommand\pubspace{3.2}
\newcommand\weirdpubspace{1.9}
\newcommand{\bburd}[1]{{\underline{\smash{#1}}}}
\newcommand{\journal}[1]{{\textbf{#1}}}
\newcommand{\pubyear}[1]{{\textbf{#1}}}
\newcommand{\dohang}{\hangindent1cm\hangafter1 }
%
\begin{longtable}{>{\everypar{\dohang}\dohang\raggedright\arraybackslash}p{0.95\textwidth}}
\hfill\\
\textit{\underline{\smash{Preprints, In Review, and In Revision}}} \hfill\\
%
\rule{0pt}{3ex}Clark, M., \bburd{G.S. Bradburd}, M. Akopyan, A. Vega, E. Rosenblum, J. Robertson.
Genetic isolation by distance underlies color pattern divergence in red-eyed treefrogs (\textit{Agalychnis callidryas}).
\emph{(in review)}.\\[\weirdpubspace em]
%
Toczydlowski, R., L. Liggins, M.R. Gaither, T.J. Anderson, R.L. Barton, 
J.T. Berg, S.G. Beskid, B. Davis, A. Delgado, E. Farrell, M. Ghoojaei, 
N. Himmelsbach, A.E. Holmes, S.R. Queeno, T. Trinh, C.A. Weyand, 
\bburd{G.S. Bradburd}, C. Riginos, R.J. Toonen, E.D. Crandall.
Lost in time and space: Poor data stewardship will hinder global genetic diversity surveillance.
\emph{(in revision)}.\\[\pubspace em]
%
\rule{0pt}{3ex}Puckett, E., S. Murphy, \bburd{G.S. Bradburd}.
Phylogeographic analysis delimits three evolutionary significant units of least chipmunks in North America and identifies unique genetic diversity within the imperiled Peñasco population.
\emph{(in review)}.\\[\pubspace em]
%
Schweizer, R.S., M.R. Jones, \bburd{G.S. Bradburd}, J.F. Storz, N. Senner, C. Wolf, Z.A. Cheviron. 
Broad concordance in the spatial distribution of adaptive and neutral genetic variation along an elevational cline in deer mice.
\emph{(in review)}.\\[\pubspace em]
%
Shaffer, H.B., E. McCartney-Melstad, P.L. Ralph, \bburd{G.S. Bradburd}, E. Lundgren, J. Vu, B. Hagerty, F. Sandmeier, C. Weitzman, C.R. Tracy.
Desert Tortoises in the genomic age: population genetics and the landscape. \underline{bioRxiv} 
(Public comment on Desert Renewable Energy Conservation).
%
\end{longtable}
%
\vspace{-1cm}
%
\begin{longtable}{>{\everypar{\dohang}\dohang\raggedright\arraybackslash}p{0.95\textwidth}}
\hfill\\
\rule{0pt}{3ex}\textit{\underline{\smash{Published and Accepted}}} \hfill (*denotes equal authorship)\\
%
%
\rule{0pt}{3ex}Hancock, Z.B., E.S. Lehmberg, \bburd{G.S. Bradburd}.
\pubyear{2021}
Neo-darwinism still haunts evolutionary theory: A modern perspective on Charlesworth, Lande, and Slatkin (1982).
\journal{Evolution} \textit{accepted}.\tabularnewline[\weirdpubspace em]
%
%
Rothstein, A., R. Knapp, \bburd{G.S. Bradburd}, D. Boiano, E.B. Rosenblum.
\pubyear{2020}.
Stepping into the past to conserve the future: archived skin swabs from extant and extinct populations inform genetic management of an endangered amphibian.
\journal{Molecular Ecology} 29 (14): 2598-2611.\tabularnewline[\pubspace em]
%
%
Fitzpatrick, S.W. \bburd{G.S. Bradburd}, C.T. Kremer, P.E. Salerno, L.M.Angeloni, W.C.Funk.
\pubyear{2020}.
Genomic and fitness consequences of genetic rescue in wild populations.
\journal{Current Biology} 30: 1-6.\tabularnewline[\weirdpubspace em]
%
%
Schweizer, R.M., J.P. Velotta, C.M. Ivy, M.R. Jones, S.M. Muir, \bburd{G.S. Bradburd}, J.F. Storz, G.R. Scott, Z.A. Cheviron.
\pubyear{2019}.
Physiological and genomic evidence that a transcription factor contributes to adaptive cardiovascular function in high-altitude deer mice.
\journal{PLoS Genetics} 15 (11), e1008420-e1008420.\\[\pubspace em]
%
%
\rule{0pt}{3ex}\bburd{Bradburd, G.S.} and P.L. Ralph.
\pubyear{2019}.
Spatial population genetics: It's about time. 
\hangindent1cm \journal{Annual Reviews in Ecology, Evolution, and Systematics} 50: 427-449.\\[1.9em]
%
%
Grieneisen, L.E., M.J.E. Charpentier, S.C. Alberts, R. Blekhman, \bburd{G.S. Bradburd}, J. Tung, E.A. Archie.
\pubyear{2019}. 
Genes, geology, and germs: gut microbiota across a primate hybrid zone are explained by site soil properties, not host species.
\journal{Proceedings of the Royal Society B} 286: 20190431.\\[\pubspace em]
%
%
\rule{0pt}{1ex}\bburd{Bradburd, G.S.}, G.M. Coop*, and P.L. Ralph*.
\pubyear{2018}. 
Inferring continuous and discrete population genetic structure across space. 
\journal{Genetics} 210: 33-52.\\[\weirdpubspace em]
%
%
Weber, J.N., \bburd{G.S. Bradburd}, Y.E. Stuart, W.E. Stutz, D.I. Bolnick.
\pubyear{2017}. 
Partitioning the effects of isolation by distance, environment, and physical barriers on genomic divergence between parapatric threespine stickleback.
\journal{Evolution} 71: 342-56.\\[\pubspace em]
%
%
Hoban, S.*, J. L. Kelley*, K. E. Lotterhos*, M. F. Antolin, \bburd{G.S. Bradburd}, D. B. Lowry, M. L. Poss, L. K. Reed, A. Storfer, M. C. Whitlock.
\pubyear{2016}.
Finding the genomic basis of local adaptation in non-model organisms: pitfalls, practical solutions, and future directions. 
\journal{American Naturalist} 188: 379-397.\\[\pubspace em]
%
%
\rule{0pt}{1ex}\bburd{Bradburd, G.S.}, P.L. Ralph, G.M. Coop.
\pubyear{2016}. 
A Spatial Framework for Understanding Population Structure and Admixture. 
\journal{PLoS Genetics} 12: e1005703.\\[\weirdpubspace em]
%
%
Hammock, B.G., S. Lesmeister, I. Flores, \bburd{G.S. Bradburd}, F.H. Hammock, S.J. Teh.
\pubyear{2016}. 
Low food availability narrows the tolerance of the copepod \textit{Eurytemora affinis} to salinity, but not to temperature. 
\journal{Estuaries and Coasts}.  39: 189-200.\\[\pubspace em]
%
%
Agrawal, A.A., A.P. Hastings, \bburd{G.S. Bradburd}, E.C. Woods, T. Z{\"u}st, J. Harvey, T. Bukovinszky.
\pubyear{2015}.
Evolution of plant growth and defense in a continental introduction. 
\journal{American Naturalist} 186: 1-15.\\[\weirdpubspace em]
%
%
Wang, I.J.* and \bburd{G.S. Bradburd}.*
\pubyear{2014}. 
Isolation by Environment. 
\journal{Molecular Ecology} 23: 5649-5662.\\[1 em]
%
%
\rule{0pt}{1ex}\bburd{Bradburd, G.S.}, P.L. Ralph, and G.M. Coop.
\pubyear{2013}. 
Disentangling the effects of geographic and ecological isolation on genetic differentiation. 
\journal{Evolution} 67: 3258-3273.\\[\weirdpubspace em]
%
%
Rejmanek, D, P. Freycon, \bburd{G.S. Bradburd}, J. Dinstell, and J. Foley.
\pubyear{2013}.
Unique strains of \textit{Anaplasma phagocytophilum} segregate among diverse questing and non-questing \textit{Ixodes} tick species in the western United States.  
\journal{Ticks and Tick-borne Diseases} 4: 482-487.\\[\pubspace em]
%
%
Rejmanek, D., \bburd{G.S. Bradburd}, and J. Foley.
\pubyear{2012}.
Molecular characterization reveals distinct genospecies of \textit{Anaplasma phagocytophilum} from diverse North American hosts.	
\journal{Journal of Medical Microbiology} 61: 204-212.\\[\pubspace em]
%
%
Near, T.J., C.M. Bossu, \bburd{G.S. Bradburd}, R.L. Carlson, R.C. Harrington, P.R. Hollingsworth Jr., B.P. Keck, D.A. Etnier.
\pubyear{2011}. 
Phylogeny and temporal diversification of darters (\textit{Percidae: Etheostomatinae}).  
\journal{Systematic Biology} 60: 565-595.
%
%
%\end{etaremune}
%
\end{longtable}
%
\section*{SOFTWARE}
\vspace{-0.6cm}
\rule{470pt}{0.4pt}
\begin{tabular}{>{\everypar{\hangindent1cm}}p{0.975\textwidth}}
\hfill\\
%
Bradburd, G.S. (2018). \textbf{conStruct}. A genetic clustering method that accounts for isolation by distance. R package version v1.0.3.
\href{https://cran.r-project.org/web/packages/conStruct/index.html}{https://cran.r-project.org/web/packages/conStruct/index.html}\\
%
\vspace{-0.1cm}
%
Bradburd, G.S. (2016). \textbf{SpaceMix}. A program for estimating geogenetic maps from allele frequency data. R package version 0.13.
\href{https://github.com/gbradburd/SpaceMix}{https://github.com/gbradburd/SpaceMix}\\
%
\vspace{-0.1cm}
%
Bradburd, G.S. (2013). \textbf{BEDASSLE}. Bayesian Estimation of Differentiation in Alleles by Spatial Structure and Local Ecology. 
R package version 1.5. \href{https://cran.r-project.org/web/packages/BEDASSLE}{https://cran.r-project.org/web/packages/BEDASSLE}\\
%
\end{tabular}
%
\section*{MENTORSHIP \& ADVISING}
\vspace{-0.6cm}
\rule{470pt}{0.4pt}
\hfill\\
\vspace{-0.9cm}
\begin{longtable}{>{\everypar{\hangindent1cm}}p{0.8\textwidth}p{0.15\textwidth}}
%
\textbf{\underline{Postdocs}}\\
\rule{0pt}{3ex}Dr. Zach Hancock & \hfill 2021-present\\
Dr. Leonard Jones & \hfill 2021-present\\
Dr. Bob Week & \hfill 2020-present\\
Dr. Rachel Toczydlowski & \hfill 2019-present\\
Dr. Matteo Tomasini & \hfill 2019-2021\\
\hspace{0.5cm} \textit{current position}: Postdoctoral Researcher, University of Gothenburg \\
Dr. Kelsey Yule & \hfill 2018-2019\\
\hspace{0.5cm} \textit{current position}: Project Manager of the NEON Biorepository, Arizona State University\\
Dr. Emily Puckett & \hfill 2017 - 2018\\
\hspace{0.5cm}\textit{current position}: Asst. Professor of Biological Sciences, U. Tennessee, Memphis\\
%
\\
%
\textbf{\underline{\smash{Graduate Students}}}\\
\rule{0pt}{3ex}Meaghan Clark & \hfill 2019-present\\
%
\\
%
\textbf{\underline{Graduate student committees}}\\
\rule{0pt}{3ex}Kyle Teller \hspace{0.25cm}(PhD, Integrated Applied Math, University of New Hampshire) &\hfill 2020-present\\
Bruce Martin \hspace{0.25cm}(PhD, Plant Biology, MSU) &\hfill 2019-present\\
Michael Foisy \hspace{0.25cm}(Msc, Plant Biology, MSU) &\hfill 2019-2020\\
Sara Hugentobler \hspace{0.25cm}(PhD, Integrative Biology, MSU) &\hfill 2019-present\\
Devin Lake \hspace{0.25cm}(PhD, Integrative Biology, MSU) &\hfill 2019-present\\
Jason Olsen \hspace{0.25cm}(PhD, Plant Biology, MSU) &\hfill 2018-present\\
Viviana Ortiz Londono \hspace{0.25cm}(PhD, Plant Pathology, MSU) &\hfill 2018-present\\
Miranda Wade \hspace{0.25cm}(PhD, Integrative Biology, MSU) &\hfill 2018-present\\
Ava Garrison \hspace{0.25cm}(PhD, Plant Biology, MSU) &\hfill 2017-present\\
Dr. Emily Dolson \hspace{0.25cm}(PhD, Computer Science and Engineering, MSU) &\hfill 2017-2019\\
%
%
%\textbf{\underline{\smash{Undergraduates}}}\\
%\rule{0pt}{3ex}Lindsay Guare & \hfill 2017 - 2019\\
%Sarah Frocillo & \hfill 2017 - 2018\\
%Nick Stants & \hfill 2017 - 2018\\
\end{longtable}
%
%\newpage
\section*{HONORS, AWARDS, \& FELLOWSHIPS}
\vspace{-0.6cm}
\rule{470pt}{0.4pt}
%
\begin{tabular}{>{\everypar{\hangindent1cm}}p{0.85\textwidth}p{0.1\textwidth}}
\hfill\\
\textbf{Fitch Award Finalist}, Society for Molecular Biology and Evolution (\$2000) & \hfill 2015\\
\textbf{Hamilton Award Finalist}, Society for the Study of Evolution (\$500) & \hfill 2015\\
\textbf{XSEDE SuperComputing Resource Allocation}, XSEDE (50,000 SUs) & \hfill 2014-05\\
\textbf{Center for Population Biology Research Fellowship} CPB, UC Davis (\$4,000) & \hfill 2011-13\\
%\textbf{Center for Population Biology Research Fellowship} CPB, UC Davis (\$2,000) & \hfill 2011-12\\
\textbf{NSF Graduate Research Fellow} National Science Foundation (\$90,000) & \hfill 2010-12\\
\textbf{Graduate Scholars Fellowship} in Population Biology, UC Davis (\$30,000) & \hfill  2009-10\\
%\hspace{4.5mm}Awarded in recognition of outstanding academic record and promise of productive scholarship\\
%\textbf{Environmental Internship Award} Yale University (\$6,000) & \hfill 2006-7\\ 
%\textbf{Richter Fellowship} Yale University (\$1,000)  & \hfill 2007\\
%\textbf{Mellon Forum Research Grant} Yale University (\$500) & \hfill 2007\\
%\textbf{Environmental Internship Award} Yale University (\$2,000) & \hfill 2006\\
\end{tabular}
%
\section*{SERVICE \& LEADERSHIP}
\vspace{-0.6cm}
\rule{470pt}{0.4pt}
\begin{tabular}{>{\everypar{\hangindent1cm}}p{0.8\textwidth}p{0.15\textwidth}}
\hfill\\
\textit{\underline{\smash{National/International Service}}}\\
\rule{0pt}{3ex}\textbf{Organizing Committee} Midwest Population Genetics Conference & \hfill 2016-present\\
%
\textbf{Course Instructor} UCLA/La Kretz Center Workshop in Conservation Genomics & \hfill 2014, 16, 19 \\
%
\textbf{Judge} Hamilton Award (Society for the Study of Evolution ) & \hfill 2017, 19 \\
%
\textbf{Reviewer} including: 
PNAS;
Ecology Letters;
Evolution Letters;
Genetics;
Evolution;
Systematic Biology; 
Molecular Biology and Evolution;
Molecular Ecology;
Proceedings of the Royal Society B;
American Journal of Human Genetics;
Heredity;
Molecular Ecology Resources;
Methods in Ecology and Evolution;
Molecular Phylogenetics and Evolution;
New Phytologist;
Bioinformatics;
Ecology
Ecological Monographs\\
%Systematics and Biodiversity;
%NSERC
\textbf{Society Memberships}
Society for the Study of Evolution (SSE); 
%Society for Molecular Biology and Evolution (SMBE); 
American Society of Naturalists (ASN);
Genetics Society of America (GSA)\\
%Diversity and Distributions;
%Integrative and Comparative Biology;
%Nature - Scientific Reports;
%Genes, Genomes, Genetics (G3);
%Ecology and Evolution;
%Australian National University
\vspace{0.3cm}
%
%
\textit{\underline{\smash{Institutional Service}}}\\
\rule{0pt}{3ex}Diversity, Equity, and Inclusion Committee (Dept. Integrative Biology) & \hfill 2019-present\\
%
Co-founder, Diversity, Equity, and Inclusion Reading Group (Dept. Integrative Biology) & \hfill 2019-present\\
%
Co-chair, Presidential Postdoc Fellowship Committee (EEB Program) & \hfill 2020-present\\
%
Seminar Committee (Dept. Integrative Biology) & \hfill 2020-present\\
%
Chair Search Committee (Dept. Integrative Biology) & \hfill 2020-present\\
%
Seminar Committee (EEB Program) & \hfill 2019-2020\\
%
Director Search Committee (EEB Program) & \hfill 2019-2020\\
%
\emph{Ad hoc} Space Committee (EEB Program) & \hfill 2019-2020\\
%
Strategic Hiring Planning Committee (Dept. Integrative Biology) & \hfill 2018-2019\\
%
%Curriculum Planning Committee (Intro to Evolution - IBio 445) & \hfill 2016-17\\
%
Student Representative, Population Biology Grad Group Steering Committee & \hfill 2013-14\\
%
Chair of Student Research Symposium, Center for Population Biology/Pop Bio Grad Group & \hfill 2013\\
%
Course Organizer, The Bodega Bay Applied Phylogenetics Workshop & \hfill 2012-13\\
%
Chair, UC Davis Non-Model Genomics Workshop Committee & \hfill 2010\\
%
Graduate Student Mentor & \hfill 2009-2015\\
%
\end{tabular}
%
\section*{TEACHING}
\vspace{-0.6cm}
\rule{470pt}{0.4pt}
%
\begin{tabular}{>{\everypar{\hangindent1cm}}p{0.60\textwidth}p{0.15\textwidth}p{0.175\textwidth}}
\hfill\\
Evolutionary Biology (IBio 849) & Instructor & \hfill 2018-present \\
Intro to Statistical Methods in Ecology/Evolution (IBio 830) & Instructor & \hfill 2017-present \\
Evolution Discussion Group (IBio 895) & Instructor & \hfill 2017-present \\
%UCLA/La Kretz Center Workshop in Conservation Genomics & Instructor & \hfill 2014, 16, 19 \\
EVE100 Introduction to Evolution & TA & \hfill 2015	\\
EVE103 Macroevolution, Speciation, and Phylogeny (Lecture, Lab) & TA & \hfill 2014 \\
Herpetology (Field, Lab, Lecture) & TA & \hfill 2012\\
%Bodega Bay Applied Phylogenetics Workshop & TA & \hfill 2012-13\\
\end{tabular}
%
\section*{INVITED SEMINARS}
\vspace{-0.6cm}
\rule{470pt}{0.4pt}
%
\begin{longtable}{>{\everypar{\hangindent1cm}}p{0.4\textwidth}p{0.4\textwidth}p{0.125\textwidth}}
%
%North Dakota State University & Dept.\ of Biological Sciences & \hfill 2021 \\
%
University of St. Andrews & Centre of Ecol. and Env. Modelling & \hfill 2021 \\
%
Cornell University & Dept.\ of Computational Biology & \hfill 2020 \\
%
Western Michigan University & Dept.\ of Biological Sciences & \hfill 2020 \\
%
Ohio State University & Dept.\ of Evolution, Ecology, Organismal Bio. & \hfill 2019 \\
%
University of Kentucky & Dept.\ of Biology & \hfill 2019 \\
%
Ecological Society of America & Landscape Genetics (student-organized) & \hfill 2018 \\
%
Indiana University & Dept.\ of Biology/Precision Health Initiative & \hfill 2018 \\
%
Penn State University & Ecology Seminar Series & \hfill 2018 \\
%
University of Toronto (St. George) & Dept.\ of Ecology and Evolutionary Biology & \hfill 2018 \\
%
University of Toronto (Mississauga) & Dept.\ of Biology & \hfill 2017 \\
%
University of Chicago & Dept.\ of Ecology and Evolution & \hfill 2017 \\
%
Michigan State University & Kellogg Biological Station & \hfill 2017 \\
%
University of Michigan & Center for Statistical Genetics & \hfill 2016 \\
%
University of Wyoming & Dept.\ of Botany & \hfill 2016 \\
%
University of Colorado Boulder & Dept.\ Ecology and Evolutionary Biology & \hfill 2016 \\
%
University of San Francisco & Dept.\ of Biology & \hfill 2016 \\
%
University of California Berkeley & Museum of Vertebrate Zoology & \hfill 2015 \\
%
Michigan State University & Ecology, Evolution, and Behavior & \hfill 2015 \\
%
Stanford University & Dept.\ Genetics and Biology & \hfill 2015 \\
%
University of Southern California & Molecular and Computational Biology & \hfill 2014 \\
%
University of Montana & Division of Biological Sciences & \hfill 2014 \\
%
University of Minnesota & Dept.\ of Plant Biology & \hfill 2014 \\
%
Sonoma State University & Biology Dept.\ & \hfill 2014 \\
%
University of Texas, Austin & Dept.\ Integrative Biology & \hfill 2013 \\
%
\end{longtable}
%
\section*{CONTRIBUTED PRESENTATIONS}
\vspace{-0.6cm}
\rule{470pt}{0.4pt}
\begin{tabular}{>{\everypar{\hangindent1cm}}p{0.75\textwidth}p{0.2\textwidth}}
\hfill\\
Society for the Study of Evolution & \hfill 2012-13,15,17,19\\
Population, Evolutionary, Quantitative Genetics Meeting & \hfill 2018\\
Midwestern Population Genetics Meeting & \hfill 2016\\
American Society of Naturalists & \hfill 2016,18\\
Society for Molecular Biology and Evolution & \hfill 2015\\
%International Association of Landscape Ecology & \hfill 2015\\
Bay Area Population Genomics (BAPG) & \hfill 2014\\
%CPB Dept. Seminar & \hfill 2011,13\\
%E\&EB Senior Symposium, Yale University & \hfill 2008\\
\end{tabular}

\end{document}

%%%%%%%%%%%%%%
%%%%%%%%%%%%%%
%%%%%%%%%%%%%%
%%%%%%%%%%%%%%
%%%%%%%%%%%%%%
%%%%%%%%%%%%%%

\section*{PREVIOUS APPOINTMENTS}
\vspace{-0.6cm}
\rule{470pt}{0.4pt}
\begin{tabular}{>{\everypar{\hangindent1cm}}p{0.8\textwidth}p{0.15\textwidth}}
\hfill\\
\textbf{Research Assistant, Prof. Thomas Near, Yale University} & \hfill 2008-09 \\ 
\hspace{4.5mm}Carried out research on phylogeography of North American fishes & \\
%
\textbf{Technician, Yale Peabody Museum of Natural History} & \hfill 2007-08 \\ 
\hspace{4.5mm}Identified and sorted vertebrate specimens&\\ 
%
\textbf{Lab Technician, Marine Biology Lab, Prof. Leo Buss} & \hfill 2005 \\ 
\hspace{4.5mm}Maintained populations of \textit{Hydractinia echinata} and \textit{Nematostella vectensis}&
\end{tabular}



\section*{FIELDWORK}
\vspace{-0.6cm}
\rule{470pt}{0.4pt}
\begin{tabular}{>{\everypar{\hangindent1cm}}p{0.8\textwidth}p{0.15\textwidth}}
\hfill\\
\textbf{Abaco Islands (Bahamas)} & \hfill 2013\\
\hspace{4.5mm}Arthropod surveys and sampling; Anole sampling.\\
%
\textbf{Northern California} & \hfill 2010-2012\\
\hspace{4.5mm}Sampled tissue and ectoparasites from lizards and small rodents.\\
%
\textbf{American Southeast and Midwest} & \hfill 2008\\ 
\hspace{4.5mm}Sampled freshwater fish throughout the Mississippi Drainage.\\
%
\textbf{Costa Rica}  & \hfill  2007\\ 
\hspace{4.5mm}Collected reptiles and amphibians in the rainforest for the Yale Peabody Museum.\\ 
\hspace{4.5mm}Developed skills in collecting and preserving herpetological specimens.\\
\hspace{4.5mm}Monitored status of the Variable Harlequin Toad (\textit{Atelopus varius}).\\
%
\textbf{Suriname}  & \hfill  2006\\
\hspace{4.5mm}Collected birds in coastal and southern Suriname for the Yale Peabody Museum.\\ 
\hspace{4.5mm}Developed skills in collecting (mist-netting, shooting) and preserving ornithological specimens (anatomicals, skeletons, skins).\\ 
\end{tabular}





\section{\underline{LANGUAGES}}
\textbf{Spanish}: proficient speaking and reading\\
\textbf{Sranan Tongo (Taki-taki)}: fair, but only in the context of collecting specimens in the field

%\end{resume}


%GRAVEYARD
\textbf{Bradburd, G.S.}, P.L. Ralph, and G.M. Coop. Disentangling the effects of geographic and ecological isolation on genetic differentiation. Evolution 67(11):3258-3273.
%
Rejmanek, D, P. Freycon, \textbf{G.S. Bradburd}, J. Dinstell, and J. Foley.  Unique strains of \textit{Anaplasma phagocytophilum} segregate among diverse questing and non-questing \textit{Ixodes} tick species in the western United States.  Ticks and Tick-borne Diseases 4(6):482-487.
%
Rejmanek, D., \textbf{G.S. Bradburd}, and J. Foley.  2012.  Molecular characterization reveals	distinct genospecies of \textit{Anaplasma phagocytophilum} from diverse North American hosts.	Journal of Medical Microbiology. 61(2):204-212.
%
Near, T.J., C.M. Bossu, \textbf{G.S. Bradburd}, R.L. Carlson, R.C. Harrington, P.R. Hollingsworth	Jr., B.P. Keck, D.A. Etnier. 2011.  Phylogeny and temporal diversification of darters (\textit{Percidae: Etheostomatinae}).  Systematic Biology 60(5):565-595.


%\cfoot{}  
%\pagestyle{fancy} 
%\newenvironment{myindentpar}[1]%
%   {\begin{list}{}%
%             {\setlength{\leftmargin}{#1}}%
%             \item[]}
%    {\end{list}}

%\textbf{Relevant Coursework}
%\begin{myindentpar}{6mm}
%{UC Davis: Core Principles in Population Biology, Mathematical Methods in Population Biology, Macroevolution, Principles of Biological Data %Analysis}\vspace{3mm}

%Yale: General Chemistry (w/ lab), General Physics, General Ecology, Evolutionary Biology, Diversity of Life, Molecular Systematics (intensive lab), %Genetics, Conservation Genetics, Phylogenetic Methods and Approaches, Biology of Terrestrial Arthropods (w/ lab), Ornithology (w/ lab), %Introductory Geoscience, Petrology and Mineralogy, Natural Resources and Sustainability
%\end{myindentpar}

\textbf{Integrative Biology Fellowship} UT Austin, 2009-'10 (declined)\\ \vspace{0.3mm}
\hspace{4.5mm}Awarded to top applicants in the UT system\\
\textbf{Dorothy Clark Lettice Fellowship} Kansas University, 2009-'10 (declined)\\  \vspace{0.3mm}
\hspace{4.5mm}Awarded in recognition of high academic ability\\

%\section{\underline{VOLUNTEER AND LEADERSHIP EXPERIENCE}}
%\textbf{Yale Freshman Outdoor Orientation Trips (FOOT) Leader} 2005-'07\\ \vspace{0.3mm}
%\hspace{4.5mm}Led 4-6 day trips in the wilderness with 8-10 incoming Yale freshmen\\ \vspace{0.3mm}
%\hspace{4.5mm}Trained and certified in Wilderness First Aid and CPR\\
%\textbf{FOOT Public Relations Core Head} 2006-'07\\ \vspace{0.3mm}
%\hspace{4.5mm}Organized all public relations initiatives for the FOOT program, including sponsorship bids, new\\ \vspace{0.3mm} 
%\hspace{4.5mm}leader recruitment, and advertising to incoming freshmen\\ 
%\textbf{Volunteer Income Tax Assistance Site Coordinator, New Haven} 2005-'07\\ \vspace{0.3mm}
%\hspace{4.5mm}Coordinated an office and trained volunteers to administer Earned Income Tax Credit to local\\ \vspace{0.3mm}
%\hspace{4.5mm}individuals and families\\ 
%\textbf{Local Director of Client Services, National Student Partnerships, New Haven} 2004-'06\\ \vspace{0.3mm}
%\hspace{4.5mm}Worked with socioeconomically disadvantaged clients to navigate social bureaucracy and find\\ \vspace{0.3mm}
%\hspace{4.5mm}employment, affordable housing, and public benefits\\ \vspace{0.3mm}
%\textbf{Community Health Educator, New Haven} 2005-'06\\ \vspace{0.3mm}
%\hspace{4.5mm}Taught in local schools to provide health education, specifically in communication skills with\\ \vspace{0.3mm}
%\hspace{4.5mm}reference to substance and relationship abuse, nutrition, and sexual health\\
%\textbf{DEMOS Classroom Volunteer, Lincoln-Bassett Elementary School, New Haven} 2004-'05\\ \vspace{0.3mm}
%\hspace{4.5mm}Demonstrated basic scientific principles in a fun and accessible way\\ 
%\textbf{Bradburd, G.S.}, P.L. Ralph, G.M. Coop.   Disentangling the effects of geographic and ecological isolation on genetic differentiation.  Society for the Study of Evolution. 2013. Snowbird, Utah.\

%\textbf{Bradburd, G.S.}  P.L. Ralph, B.Moore, G.M. Coop. A Bayesian method for estimating genetic differentiation due to isolation %by geographic and ecological distance.  Society for the Study of Evolution. 2013. Ottawa, Canada.\

%\textbf{Bradburd, G.S.} Retrieving lost evolutionary history from an extinct darter species.  Ecology and Evolutionary Biology Senior Symposium, Yale University, 2008.

%\textbf{Bradburd, G.S.}, B.Moore, G.M. Coop, J.E. Foley. Molecular ecology of two tick-borne pathogens in California.  Center for Population Biology Departmental Seminar. 2011. UC Davis.\

%\section*{INVITED TALKS}
%\vspace{-0.6cm}
%\rule{470pt}{0.4pt}
%%
%\begin{tabular}{>{\everypar{\hangindent1cm}}p{0.85\textwidth}p{0.1\textwidth}}
%\hfill\\
%%
%\textbf{Bradburd, G.S.} Patterns of Spatial Genetic Variation. & \hfill 2015 \\
%\hspace{1cm} Michigan State University.\\
%%
%\textbf{Bradburd, G.S.}, P. Ralph, G. Coop. The Geography of Genetic Structure and Admixture. & \hfill 2015 \\
%\hspace{1cm} Stanford University.\\
%%
%\textbf{Bradburd, G.S.}, P. Ralph, G. Coop. A new approach to the geography of genetic structure. & \hfill 2014 \\
%\hspace{1cm} University of Southern California.\\
%%
%\textbf{Bradburd, G.S.}, P. Ralph, G. Coop. The spatial context of genetic admixture. & \hfill 2014 \\
%\hspace{1cm} University of Montana.\\
%%
%\textbf{Bradburd, G.S.}, P. Ralph, G. Coop.  A spatial framework for studying genetic differentiation.  & \hfill 2014 \\
%\hspace{1cm} University of Minnesota.\\
%%
%\textbf{Bradburd, G.S.} The effects of geographic and ecological isolation on genetic differentiation. & \hfill 2014 \\
%\hspace{1cm} Sonoma State University. \\
%%
%\textbf{Bradburd, G.S.} \textit{BEDASSLING} your data: background and functions of the R package \textit{BEDASSLE}.  & \hfill 2013 \\
%\hspace{1cm} University of Texas, Austin\\
%\end{tabular}